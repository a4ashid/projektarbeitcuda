\documentclass[twocolumn]{IEEEtran}
%\usepackage{epsf}
%\usepackage{./fit}
%\usepackage{./forms}
\usepackage[ansinew]{inputenc}

\def\BibTeX{{\rm B\kern-.05em{\sc i\kern-.025em b}
\kern-.08em T\kern-.1667em\lower.7ex\hbox{E}\kern-.125emX}}

\setcounter{page}{1}
\setlength\arraycolsep{1pt}

\newcommand{\fMm}{{\bf M}_{\rm m}}
\newcommand{\fMl}{{\bf M}_{\lambda}}

\begin{document}

\title{Wissenschaftliches Programmieren mit CUDA}
%
\thanks{\hrule}%
\thanks{Eingereicht am 05.02.2009}%
\thanks{Daniel Klimeck (Matnr. 6345768), }
\thanks{Christian Renneke (Matnr. 6257603) }
\thanks{Universit�t Paderborn}
\thanks{FG Theoretische Elektrotechnik}
\thanks{Warburger Str. 100}
\thanks{33098 Paderborn}
\thanks{Germany}
\thanks{Email: Klimeck@tet.upb.de / Renneke@tet.upb.de}


\author{Daniel Klimeck und Christian Renneke}

\markboth{Ausarbeitung zum Projekt} {\it Wissenschaftliches Programmieren mit CUDA }%


\maketitle

\begin{abstract}
Das Projekt "`Wissenschaftliches Programmieren mit CUDA"' besch�ftigte sich mit der Berechnung von Simulationen aus dem Bereich der theoretischen Elektrotechnik. Das Verarbeiten der Algorithmen sollte mit Hilfe der Rechenleistung einer Grafikkarte beschleunigt werden, um so die Simulationszeiten zu reduzieren. Die Implementierung erfolgte mittels CUDA ( Compute Unified Device Architecture ) von dem Unternehmen NVIDIA. Dabei handelt es sich um ein SDK ( Software Development Kit ) was die Programmierung der Grafikprozessoren von NVIDIA erm�glicht.
\end{abstract}


\section{Einf�hrung}
Da bei der Verbesserung von Prozessoren die Erh�hung der Taktfrequenz an ihre wirtschaftlichen Grenzen st��t setzen die f�hrenden Unternehmen auf parallele Verarbeitung. So werden aktuell vier Kerne in einer CPU genutzt, um mehr Rechenleistung zu erhalten.\\
Dieser Trend ist bei Grafikkarten bereits deutlich l�nger zu beobachten. Deren Prozessoren sind darauf ausgelegt immer �hnliche Operationen auf gro�e Datenmengen anzuwenden. Dies l�sst sich sehr gut parallelisieren, so dass in Grafikkarten meist mehrere Hundert parallele Prozessoren arbeiten und dies resultiert in einer deutlich h�heren Rechenleistung. Um diese hohe Rechenleistung nun auch f�r andere Zwecke als zur Bildverarbeitung nutzen zu k�nnen hat NVIDIA ein SDK herausgebracht, dass es erm�glicht C- oder Fortran- Code auf der GPU auszuf�hren. Diese Technologie ist f�r alle von Interesse, die gro�e Datenmengen auf �hnliche Weise verarbeiten und berechnen m�ssen, wie es zum Beispiel bei vielen Simulationen der Fall ist. \\
Daher sollte in diese Projektarbeit erarbeitet werden wie gut sich die Rechenleistung der GPUs nutzen l�sst, um typische Simulationsaufgaben der theoretischen Elektrotechnik schneller auszuf�hren. Dabei sollte ebenfalls untersucht werden, ob die Nutzung der Grafikprozessoren Genauigkeitsverkuste mit sich bringt und die Programmierung von CUDA erlernt werden.


------------------\\
GPUs und parallele berechnungen - Vorteile - nachteile - CUDA als Programmierumgebung - erwartete ergebnisse


\section{CUDA}
Die \textbf{C}ompute \textbf{U}nified \textbf{D}evice \textbf{A}rchitecture des Grafikkartenherstellers NVIDIA ist eine Schnittstelle, die es erm�glicht die Rechenleistung der Grafikprozessoren f�r allgemeine Zwecke zu gebrauchen. So wird von dieser API eine Programmierumgebung zur Verf�gung gestellt, mit der man C-Code auf Grafikprozessoren ausf�hren kann. Diese haben 
Programmierung unter C\\
Aufruf in Matlab �der MEX als Schnittstelle\\
Aufbau der Grafikkarten\\

\section{Simulationsbeispiel }
\subsection{Leapfrog-Algorithmus}

\subsection{Umsetzung in CUDA}
- Beispielcode
\section{Ergebnisse DANIEL}
Vorstellung der Beispiele, Bragg Reflke
Diagramm
\section{Zusammenfassung DANIEL}

-Lohnt es sich f�r gro�e Beispiele
-Kostenreduzierend\\
\bibliographystyle{IEEE}

\begin{thebibliography}{2}

\bibitem{Hameyer}
J. Driesen, R. Belmans, K. Hameyer, ``Adaptive relaxation algorithms
for thermo-electromagnetic FEM problems'', IEEE Trans. on Magnetics,
Vol.\ 35, No.\ 3, 1999, pp.~1622-1625.

\bibitem{Kost}
L. J\"anicke, A. Kost, ``Convergence properties of of the Newton-Raphson
Method for nonlinear problems'', in Proc.  $\rm XI^{th}$ IEEE COMPUMAG,
Rio de Janeiro, Brazil, PF3-6, 1997, pp.~585-586.


\end{thebibliography}

\end{document}
